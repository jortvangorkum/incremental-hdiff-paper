\chapter{Concise, Type-Safe, and Efficient Structural Diffing}

\section{Abstract}
A structural diffing algorithm compares two pieces of tree- shaped data and computes their difference. Existing structural diffing algorithms either produce concise patches or en- sure type safety, but never both. We present a new structural diffing algorithm called \textit{truediff} that achieves both properties by treating subtrees as mutable, yet linearly typed resources. Mutation is required to derive concise patches that only mention changed nodes, but, in contrast to prior work, true- diff guarantees all intermediate trees are well-typed. We formalize type safety, prove \textit{truediff} has linear run time, and evaluate its performance and the conciseness of the derived patches empirically for real-world Python documents. While \textit{truediff} ensures type safety, the size of its patches is on par with Gumtree, a popular untyped diffing implementation. Regardless, \textit{truediff} outperforms Gumtree and a typed diffing implementation by an order of magnitude.

\section{Introduction}
\begin{itemize}
    \item Most existing structural diffing algorithms follow an approach pioneered by Chawathe et al. Their approach represents structural patches as edit scripts, which convert a source tree into a target tree through consecutive destructive updates.
    \item Miraldo and Swierstra recently presented a new type-safe diffing algorithm. While this approach can capture moved subtrees, it suffers one major problem:
    \begin{itemize}
        \item The size of the patch is proportional to the size of the input trees and must mention many unchanged nodes.
        \item Consequently, any subsequent transmission or processing of the patch will essentially require a full tree traversal, even for small changes.
    \end{itemize}
    \item Like Chawathe et al., we consider trees as mutable data and use URIs to refer to changed nodes directly. However, unlike them, we consider subtrees as linear resources and target a novel linearly typed edit script language called truechange.
    \item The type system of truechange ensures that: 
    \begin{itemize}
        \item each edit operation yields a well-typed tree (possibly containing holes), 
        \item the final tree is well-typed and has no holes, and 
        \item all detached subtrees are reattached or deleted.
    \end{itemize}
    \item We use a novel strategy in \textit{truediff} for identifying reusable subtrees that should be moved. 
    \begin{itemize}
        \item In a first step, we identify candidates as those trees that are equivalent except for literal values.
        \item In a second step, we select an exact copy from the candidates if possible or otherwise adapt an imperfect candidate if needed.
    \end{itemize}
    \item Summary:
    \begin{itemize}
        \item We introduce a linearly typed edit script language \textit{truechange} that treats subtrees as resources, and we prove well-typed edit scripts yield well-typed trees.
        \item We develop a structural diffing algorithm \textit{truediff} that yields concise and type-safe edit scripts and we prove it runs in linear time. 
        \item We implement \textit{truediff} in Scala and provide bindings for ANTLR, treesitter and Gumtree.
        \item We evaluate the conciseness and performance of \textit{truediff} and the applicability for incremental computing.
    \end{itemize}
\end{itemize}

\section{Linearly Typed Edit Scripts by Example}
\begin{itemize}
    \item A concise representation of structural patches should only mention changed nodes, such that the patch is proportional in size to the change. Therefore, \textit{truechange} uses URIs to refer to changed nodes directly.
\end{itemize}

\section{\textit{truechange}: Linearly Typed Edit Scripts}
\begin{itemize}
    \item Like previous untyped edit script languages, \textit{truechange} edits describe destructive updates of the source tree, so that only changed nodes need to be mentioned. However, like previous type-safe edit script languages, \textit{truechange} guarantees that each edit operation yields a well-typed tree
    \item A \textit{truechange} edit script is a sequence of detach, attach, load, unload, and update operations.
\end{itemize}

\newpage
\section{\textit{truediff}: Type-Safe Structural Diffing}
\begin{itemize}
    \item To compute the difference between a source tree this and a target tree that, \textit{truediff} operates in four steps.
    \begin{itemize}
        \item Prepare subtree equivalence relations
        \item Find reuse candidates
        \item Select reuse candidates
        \item Compute edit script
    \end{itemize}
    \item Our algorithm \textit{truediff} generalizes this idea and uses two equivalence relations, both encoded through cryptographic hashes.
    \begin{itemize}
        \item The first equivalence relation identifies reuse candidates.
        \item The second equivalence relation identifies preferred trees among the reuse candidates.
    \end{itemize}
    \item We found that using \textbf{structural equivalence} to identify candidates and \textbf{literal equivalence} to select preferred candidates yields very concise edit scripts.
\end{itemize}

\section{Evaluation}
\begin{itemize}
    \item After a code change, we reparse the source file, use \textit{truediff} to obtain an edit script, and then process the edits to trigger updates in the incrementally maintained Datalog database.
\end{itemize}

\section{Related Work}
\begin{itemize}
    \item In designing truechange, we replaced the move operation with separate detach and attach operations. This change enabled us to formalize a type system for edit scripts that ensures all intermediate trees are well-typed.
    \item In the design of \textit{truediff}, we do not rely on similarity scores but instead designate reusable trees based on structural and literal subtree equivalences.
    \item But hdiff has three main limitations that motivated the development of \textit{truediff}. 
    \begin{itemize}
        \item First, hdiff assumes isomorphic subtrees are equal and can thus be shared. However, many applications consider the context surrounding a subtree (its parent, neighbors, etc.), which precludes sharing.
        \item Second, the size of the patch computed by hdiff is proportional to the size of the source and target trees, despite supporting move edits.
        \item And third, the running time of hdiff was unsatisfactory and precluded its application in incremental computing.
    \end{itemize}
    \item Tree-sitter is a modern incremental parsing implementation based on the incremental LR parsing algorithm by Wagner and Graham. This algorithm tries to reuse subtrees of the previous AST, but only if their relative position has not changed.
\end{itemize}

\newpage
\section{Conclusion}
We presented \textit{truediff}, an efficient structural diffing algorithm that yields concise and type-safe patches. In comparing trees, \textit{truediff} treats subtrees as mutable, yet linearly typed resources. As such, subtrees can only be attached once and slots in parent nodes can only be filled when they are empty. We capture these invariants in a new linearly typed edit script language truechange that we introduced and for which we proved type safety. To generate concise patches, \textit{truediff} follows a novel strategy for identifying reusable subtrees: While other approaches rely on similarity scores, \textit{truediff} uses efficiently computable equivalence classes to find and select reuse candidates. As our empirical evaluation demonstrates, this strategy enables \textit{truediff} to deliver concise patches on par with the state of the art, while being an order of magnitude faster. We have adopted \textit{truediff} to drive an incremental program analysis framework, which shows that \textit{truediff} is useful in practice.