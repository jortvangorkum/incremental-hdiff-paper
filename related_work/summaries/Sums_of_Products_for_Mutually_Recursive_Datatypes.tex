\chapter{Sums of Products for Mutually Recursive Datatypes}

\section{Abstract}
Generic programming for mutually recursive families of datatypes is hard. On the other hand, most interesting abstract syntax trees are described by a mutually recursive family of datatypes. We could give up on using that mutually recursive structure, but then we lose the ability to use those generic operations which take advantage of that same structure. We present a new approach to generic programming that uses modern Haskell features to handle mutually recursive families with explicit sum-of-products structure. This additional structure allows us to remove much of the complexity previously associated with generic programming over these types.

\section{Introduction}
\begin{itemize}
    \item The novelty in our work is in the intersection of both the expressivity of multirec, allowing the encoding of mutually recursive families, with the convenience of the more modern generics-sop style.
    \item The availability of several libraries for generic programming witnesses the fact that there are trade-offs between expressivity, ease of use, and underlying techniques in the design of such a library.
\end{itemize}

\subsection{Explicit Recursion}
\begin{itemize}
    \item If we do not mark recursion explicitly, \textit{shallow} encodings are our sole option, where only one layer of the value is turned into a generic form.
    \item The other side of the spectrum would be the \textit{deep} representation, in which the entire value is turned into the representation that the generic library provides in one go. 
    \begin{itemize}
        \item These representations are usually more involved as they need an extra mechanism to represent recursion.
        \item A \textit{deep} encoding requires some explicit \textit{least fixpoint} combinator - usually called \textit{Fix} in Haskell.
    \end{itemize}
    \item Depending on the use case, a shallow representation might be more efficient if only part of the value needs to be inspected. On the other hand, deep representations are sometimes easier to use, since the conversion is performed in one go, and afterwards one only has to work with the constructs from the generic library.
\end{itemize}

\subsection{Sum of Products}
\begin{itemize}
    \item Most generic programming libraries build their type level descriptions out of three basic combinators
    \begin{itemize}
        \item \textit{constants}, which indicate a type is atomic and should not be expanded further;
        \item \textit{products} (usually written as $:*:$) which are used to build tuples;
        \item and \textit{sums} (usually written as $:+:$) which encode the choice between constructors.
    \end{itemize}
    \item The \texttt{generic-sop} library explicitly uses a list of lists of types,
    \begin{itemize}
        \item the outer one representing the sum
        \item and each inner one thought of as products.
    \end{itemize}
    \item The \inlinehaskell{`} sign in the code below marks the list as operating at the type level, as opposed to term-level lists which exist at run-time.
    \begin{itemize}
        \item An example of Haskell's \textit{datatype} promotion.
        \begin{haskell}
            Code_sop(Bin a) = `[`[a], `[Bin a, Bin a]]
        \end{haskell}
    \end{itemize}
    \item The \textit{representation} is mapping the \textit{codes}, of kind \inlinehaskell{`[`[*]]} into \inlinehaskell{*}. The \textit{code} can be seen as the format that the \textit{representation} must adhere to. Previously, in the pattern functor approach, the \textit{representation} was not guaranteed to have a certain structure.
\end{itemize}

\subsection{Mutually recursive datatypes}
\begin{itemize}
    \item Unfortunately, most of the generic programming approaches restrict themselves to \textit{regular} types, in which recursion always goes into the \textit{same} datatype, which is the one being defined.
    \item The syntax of many programming languages is expressed naturally using a mutually recursive family. 
    \item The motivation of our work stems from the desire of having the concise structure that \textit{codes} give to the \textit{representations}, together with the information for recursive positions in a mutually recursive setting.
\end{itemize}

\section{Background}
\subsection{GHC Generics}
\begin{itemize}
    \item Upon reflecting on the generic function, we see a number of issues.
    \begin{itemize}
        \item Most notably is the amount of boilerplate to achieve a conceptually simple task. This is a direct consequence of not having access to the \textit{sum-of-products} structure that Haskell's \inlinehaskell{data} declarations follow.
        \item A second issue is that the generic representation does not carry any information about the recursive structure of the type.
        \item Perhaps even more subtle, but also more worrying, is that we have no guarantees that the $Rep_{gen} \:\; a$ of type \textit{a} will be defined using only the supported \textit{pattern functors}. 
    \end{itemize}
\end{itemize}

\subsection{True Sums of Products}
\begin{itemize}
    \item In comparison to the \texttt{GHC Generics} implementation we see two improvements.
    \begin{itemize}
        \item We need one fewer type class, and
        \item the definition is combinator-based.
    \end{itemize}
    \item There are still downsides to this approach.
    \begin{itemize}
        \item A notable one is the need to carry constraints around.
    \end{itemize}
\end{itemize}

\section{Explicit Fix: Diving Deep and Shallow}
\begin{itemize}
    \item We have combined the insight from the \texttt{regular} library of keeping track of recursive positions with the convenience of the \texttt{generics-sop} for enforcing a specific \textit{normal form} on representations. By doing so, we were able to provide a \textit{deep} encoding for free.
\end{itemize}