\chapter{An Efficient Algorithm for Type-Safe Structural Diffing}

\section{Abstract}
Effectively computing the difference between two versions of a source file has become an indispensable part of software development. The de facto standard tool used by most version control systems is the UNIX diff utility, that compares two files on a line-by-line basis without any regard for the structure of the data stored in these files. This paper presents an alternative datatype generic algorithm for computing the difference between two values of any algebraic datatype. This algorithm maximizes sharing between the source and target trees, while still running in linear time. Finally, this paper demonstrates that by instantiating this algorithm to the Lua abstract syntax tree and mining the commit history of repositories found on GitHub, the resulting patches can often be merged automatically, even when existing technology has failed.

\section{Introduction}
\begin{itemize}
    \item A consequence of the by line granularity of the UNIX diff is it inability to identify more fine-grained changes in the objects it compares.
    \item Ideally, however, the objects under comparison should dictate the granularity of change to be considered. This is precisely the goal of structural differencing tools.
    \item In this paper we present an efficient datatype-generic algorithm to compute the difference between two elements of any mutually recursive family
    \item The \textit{diff} function computes these differences between two values of type \textit{a}, 
    \item and \textit{apply} attempts to transform one value according to the information stored in the
    \textit{Patch} provided to it.
    \item We expect certain properties of our diff and apply functions.
    \begin{itemize}
        \item The first being \textit{correctness}: the patch that \textit{diff x y} computes can be used to faithfully reproduces \textit{y} from \textit{x}.
        \[
            \forall \: x \: y \: . \: apply (diff \: x \: y) \: x \: \equiv \: Just \: y
        \]
        \item The second being \textit{preciseness}:
        \[
            \forall \: x \: y \: . \: apply (diff \: x \: x) \: y \: \equiv \: Just \: y
        \]
        \item The last being \textit{computationally efficient}: Both the \textit{diff} and \textit{apply} functions needs to be space and time efficient.
    \end{itemize}
    \item There have been several attempts at generalizing UNIX diff results to handle arbitrary datatypes, but following the same recipe: enumerate all combinations of insertions, deletions and copies that transform the source into the destination and choose the 'best' one. We argue that this design has two weaknesses when generalized to work over arbitrary types:
    \begin{itemize}
        \item The non-deterministic nature of the design makes the algorithms inefficient.
        \item There exists no canonical 'best' patch and the choice is arbitrary.
    \end{itemize}
    \item This paper explores a novel direction for differencing algorithms: rather than restricting ourselves to \textit{insertions, deletions,} and \textit{copy operations}, we allow the \textit{arbitrary reordering, duplication,} and \textit{contraction of subtrees}.
\end{itemize}

\section{Tree Diffing: A Concrete Example}
\begin{itemize}
    \item We explicitly model permutations, duplications and contractions of subtrees within our notion of \textit{change}. Where contraction here denotes the partial inverse of a duplication.
    \item The representation of a \textit{change} between two values of type \inlinehaskell{Tree23}, is given by identifying the bits and pieces that must be copied from source to destination making use of permutations and duplications where necessary.
    \item A new datatype \inlinehaskell{Tree23C} $\varphi$, enables us to annotate a value of \inlinehaskell{Tree23} with holes of $\varphi$. Therefore, \inlinehaskell{Tree23C MetaVar} represents the type of \inlinehaskell{Tree23} with holes carrying metavariables.
    \item These metavariables correspond to arbitrary trees that are \textit{common subtrees} of both the source and destination of change.
    \item We refer to a value of \inlinehaskell{Tree23C} as a \textit{context}.
    \item A \textit{change} in this setting is a pair of such contexts. The first context defines a pattern that binds some metavariables, called the \textbf{deletion context}; the second, called the \textbf{insertion context}, corresponds to the tree annotated with the metavariables that are supposed to be instantiated by the binding given by the deletion context.
    \begin{haskell}
        type Change23 !$\varphi$! = (Tree23C !$\varphi$!, Tree23C !$\varphi$!)
    \end{haskell}
    \item Applying a change is done by instantiating the metavariables in the deletion context and the insertion context:
    \begin{haskell}
        applyChange :: Change23 MetaVar -> Tree23 -> Maybe Tree23
        applyChange (d, i) x = del x >>= ins i
    \end{haskell}
    \item The changeTree23 function merely has to compute the deletion and insertion contexts
    \begin{haskell}
        changeTree23 :: Tree23 -> Tree23 -> Change23 MetaVar
        changeTree23 s d = (extract (wcs s d) s, extract (wcs s d) d)
    \end{haskell}
    \item The extract function receives an oracle and a tree. It traverses its argument tree, looking for opportunities to copy subtrees.
    \begin{haskell}
        extract :: (Tree23 -> Maybe MetaVar) -> Tree23 -> Tree23C MetaVar
        extract o t = maybe (peel t) Hole (o t)
            where peel Leaf = LeafC
                  peel (Node2 a b)   
                    = Node2C (extract o a) (extract o b)
                  peel (Node3 a b c) 
                    = Node3C (extract o a) (extract o b) (extract o c)
    \end{haskell}
    \item This iteration of the changeTree23 function has a subtle bug: not all common subtrees can be copied. In particular, we cannot copy a tree t that occurs as a subtree of the source and destination, but also appears as a subtree of another, larger common subtree.
    \item One way to solve this is to introduce an additional post-processing step that substitutes the variables that occur exclusively in the deletion or insertion context by their corresponding tree.
    \begin{haskell}
        changeTree23 :: Tree23 -> Tree23 -> Change23 MetaVar
        changeTree23 s d 
            = postprocess s d (extract (wcs s d) s) (extract (wcs s d) d)
    \end{haskell}
\end{itemize}

\subsection{Minimizing Changes}
\begin{itemize}
    \item The process of minimizing and isolating the changes starts by identifying the redundant part of the contexts. That is, the constructors that show up as a prefix in both the deletion and the insertion context.
    \item They are essentially being copied over, and we want to make this fact explicit by separating them into what we call the \textit{spine} of the patch.
    \item If a constructor is in the spine, we know it has been copied, if it shows up in a change, we know it was either deleted or inserted.
    \begin{haskell}
        type Patch23 = Tree23C (Change23 MetaVar)

        patch :: Patch23
        patch = Node3C (Hole (Hole 0, Hole 0))
                       (Hole (Node2C (Hole 0) (Hole 1)
                             , Node2C (Hole 1) (Hole 0)))
                       (Node2C (Hole (tree23ToC w, tree23ToC w'))
                               (Hole (Hole 3, Hole 3)))
    \end{haskell}
    \begin{figure}[H]
        \centering
        \includegraphics[width=.8\textwidth]{patch_tree23.png}
        \caption{Patch23 Example}
        \label{fig:patch_tree23}
    \end{figure}
    \item A patch consists in a spine with changes inside it. Figure \ref{fig:patch_tree23} illustrates a value of type Patch23, where the changes are visualized with a shaded background in the leaves of the spine.
    \item The first step to compute a patch from a change is identifying its spine.
    \item We are essentially splitting
    a monolithic change into the greatest common prefix of the insertion and deletion contexts, leaving smaller changes on the leaves of this prefix:
    \begin{haskell}
        gcp :: Tree23C var -> Tree23C var -> Tree23C (Change23 var)
        gcp LeafC LeafC = LeafC
        gcp (Node2C a1 b1) (Node2C a2 b2) 
            = Node2C (gcp a1 a2) (gcp b1 b2)
        gcp (Node3C a1 b1 c1) (Node3C a2 b2 c2)
            = Node3C (gcp a1 a2) (gcp b1 b2) (gcp c1 c2)
    \end{haskell}
    \item The greatest common prefix consumes all the possible constructors leading to disagreeing parts of the contexts where this might be too greedy.
    \item To address this problem, we go over the result from our call to \inlinehaskell{gcp}, pulling changes up the tree until each change is closed, that is, the set of variables in both contexts is identical. We call this process the closure of a patch
    \item The final diff function for \inlinehaskell{Tree23} is then defined as follows:
    \begin{haskell}
        diffTree23 :: Tree23 -> Tree23 -> Patch23
        diffTree23 s d = closure $ gcp $ changeTree23 s d
    \end{haskell}
\end{itemize}

\subsection{Defining the \inlinehaskell{wcs} for \inlinehaskell{Tree23}}

\begin{itemize}
    \item In order to have a working version of our diff algorithm for Tree23 we must provide the wcs implementation. Recall that the wcs function, \textit{which common subtree}, has type:
    \begin{haskell}
        wcs :: Tree23 -> Tree23 -> Tree23 -> Maybe MetaVar
    \end{haskell}
    \item Given a fixed \textit{s} and \textit{d}, \textit{wcs s d x} returns Just \textit{i} if \textit{x} is the $i^{th}$ subtree of
    \textit{s} and \textit{d} and Nothing if \textit{x} does not appear in \textit{s} or \textit{d}.
    \item To tackle the first issue and efficiently compare trees for equality we will be using cryptographic hash functions to construct a fixed length bitstring that uniquely identifies a tree modulo hash collisions.
    \item Said identifier will be the hash of the root of the tree, which will depend on the hash of every subtree, much like a Merkle tree
    \begin{haskell}
        merkleRoot :: Tree23 -> Digest
        merkleRoot Leaf = emptyDigest
        merkleRoot (Node2 x y)
            = hash (concat ["node2", merkleRoot x, merkleRoot y])
        merkleRoot (Node3 x y z)
            = hash (concat ["node3", merkleRoot x, merkleRoot y, merkleRoot z])
    \end{haskell}
    \item the ($\equiv$) definition above is still
    linear, we recompute the hash on every comparison. We fix this by caching the hash associated with every node of a Tree23. This is done by the decorate function
    \begin{haskell}
        data Tree23H = LeafH
                     | Node2H (Tree23H, Digest) (Tree23H, Digest)
                     | Node3H (Tree23H, Digest) 
                              (Tree23H, Digest) 
                              (Tree23H, Digest)
    \end{haskell}
    \item This enables us to define a constant time merkleRoot function, shown below, which makes the ($\equiv$) function run in constant time.
    \begin{haskell}
        merkleRoot :: Tree23H -> Digest
        merkleRoot LeafH = emptyDigest
        merkleRoot (Node2H (_, hx) (_, hy)) 
            = hash (encode "2" ++ hx ++ hy)
        merkleRoot (Node3H (_, hx) (_, hy) (_, hz))
            = hash (encode "3" ++ hx ++ hy ++ hz)
    \end{haskell}
    \item The second source of inefficiency is enumerating all possible subtrees, which can be addressed by choosing a better data structure.
    \item  Given that a Digest is just a [Word], the optimal choice for such “database” is a Trie, mapping a [Word] to a MetaVar. Trie lookups are efficient and hardly depend on the number of elements in the trie.
\end{itemize}

\section{Tree Diffing Generically}
\begin{itemize}
    \item Take the Tree23 type, its structure can be seen in a sum-of-products fashion through the Tree23SOP type given below.
    \begin{haskell}
        type Tree23SOP = `[`[]                -- Leaf
                          , `[I 0, I 0]       -- Node2
                          , `[I 0, I 0, I 0]] -- Node3
    \end{haskell}
    \item The outer list represents the choice of constructor, and packages the sum part of the datatype whereas the inner list represents the product of the fields of a given constructor.
    \item The \inlinehaskell{`} notation represents a value that has been promoted to the type level
    \item The atoms, in this case only \inlinehaskell{I 0}, represent a recursive position referencing the first type in the family.
    \item The \texttt{generics-mrsop} uses the type \inlinehaskell{Atom} to distinguish whether a field is a recursive position referencing the n-th type in the family, I n, or an opaque type, for example, \inlinehaskell{Int} or \inlinehaskell{Bool}, which are represented by \inlinehaskell{K KInt}, \inlinehaskell{K KBool}.
    \item Let us now take a mutually recursive family with more than one element and see how it is represented within the generics-mrsop framework.
    \begin{haskell}
        data Zig = Zig Int  | ZigZag Zag
        data Zag = Zag Bool | ZigZag Zig

        type ZigCodes = `[`[`[K KInt],  `[I 1]]
                         ,`[`[K KBool], `[I 0]]]
    \end{haskell}
    \item The code acts as a recipe that the representation functor must follow in order to interpret those into a type of kind *.
    \item The representation is defined by the means of n-ary sums (\textit{NS}) and products (\textit{NP}) that work by induction on the codes and one interpretation for atoms (\textit{NA}).
    \item We define the representation functor Rep as the composition of the interpretations of the different pieces:
    \begin{haskell}
        type Rep !$\varphi$! = NS (NP (NA !$\varphi$!))
    \end{haskell}
    \item Finally, we tie the recursive knot with a functor of kind \inlinehaskell{Nat -> *} that is passed as a parameter to \textit{NA} in order to interpret the recursive positions.
    \begin{haskell}
        newtype Fix (codes :: `[`[`[Atom]]]) (ix :: Nat)
            = Fix { unFix :: Rep (Fix codes) (Lkup codes ix) }
    \end{haskell}
    \item Here, \inlinehaskell{Lkup codes ix} denotes the type level lookup of the element with index \texttt{ix} within the list codes.
    \item We start defining the generic notion of context, called \inlinehaskell{Tx}. Analogously to \inlinehaskell{Tree23C}.
    \item \inlinehaskell{Tx} enables us to augment mutually recursive family with type holes.
    \item We can read \inlinehaskell{Tx codes !$\varphi$! at} as the element of the mutually recursive family \texttt{codes} indexed by \texttt{at} augmented with holes of type $\varphi$. Its definition follows:
    \begin{haskell}
        data Tx :: [[[Atom]]] -> (Atom -> *) -> Atom -> * where
            TxHole :: !$\varphi$! at -> Tx codes !$\varphi$! at
            TxOpq  :: Opq k -> Tx codes !$\varphi$! (K k)
            TxPeel :: Constr (Lkup codes i) c
                   -> NP (Tx codes !$\varphi$!) (Lkup (Lkup codes i) c)
                   -> Tx codes !$\varphi$! (I i) 
    \end{haskell}
    \item Looking at the definition of \inlinehaskell{Tx}, we see that its values consist in either a typed hole, some opaque value, or a constructor and a product of fields.
    \item The \inlinehaskell{Tx} type is, in fact, the indexed free monad over the \inlinehaskell{Rep}.
    \item Essentially, a value of type \inlinehaskell{Tx codes !$\varphi$! at} is a value of type \inlinehaskell{NA (Fix codes) at} augmented with holes of type $\varphi$.
\end{itemize}    

\subsection{Generic Representation of Changes}
\begin{itemize}
    \item With a generic notion of contexts, we can go ahead and define our generic Change type.
    \begin{haskell}
        data Change codes !$\varphi$! at = Change (Tx codes !$\varphi$! at) (Tx codes !$\varphi$! at)
    \end{haskell}
    \item The interpretation for the metavariables, \inlinehaskell{MetaVar}, now carries the integer representing the metavariable itself but also carries information to identify whether this metavariable is supposed to be instantiated by a recursive member of the family or an opaque type.
    \begin{haskell}
        data MetaVar at = MetaVar Int (NA (Const Unit) at)
    \end{haskell}
    \item The type of changes over \inlinehaskell{Tree23} can now be written using the generic representation for changes and metavariables.
    \begin{haskell}
        type ChangeTree23 = Change Tree23Code MetaVar (I 0)
    \end{haskell}
    \item We can read the type above as the type of changes over the zero-th \inlinehaskell{(I 0)} type within the mutually recursive family \inlinehaskell{Tree23Code} with values of type \inlinehaskell{MetaVar} in its holes.
\end{itemize}

\newpage
\section{Defining the Generic Oracle}
\begin{itemize}
    \item  The \textit{synthesize} function is just like a catamorphism, but we decorate the tree with the intermediate results at each node, rather than only using them to compute the final outcome. This enables us to decorate each node of a \inlinehaskell{Fix codes} with a unique identifier by running the generic decorate function, defined below.
    \begin{haskell}
        newtype AnnFix x codes i 
            = AnnFix (x i, Rep (AnnFix x codes) (Lkup codes i))

        decorate :: Fix codes i -> AnnFix (Const Digest) codes i
        decorate = synthesize authAlgebra

        merkleRoot :: AnnFix (Const Digest) codes i -> Digest
        merkleRoot (AnnFix (Const r, _)) = r
    \end{haskell}
    \item Here, \inlinehaskell{AnnFix} is the cofree comonad, used to add a label to each recursive branch of our generic trees.
    \item The \inlinehaskell{Exist} datatype simply encapsulate the \texttt{ix} type index from a \inlinehaskell{Fix codes ix} into an existential one.
\end{itemize}

\section{Merging Patches}
\begin{itemize}
    \item The merging problem is the problem of computing a new patch, $q \: // \: p$, given two patches \textit{p} and \textit{q}. It consists in a patch that contains the changes of \textit{q} adapted to work on a value that has already been modified by \textit{p}.
    \item If \textit{p} and \textit{q} are disjoint, then $p \: // \: q$ can return \textit{p} without further adaptations. Our algorithm shall merge only disjoint patches, marking all other situations as a conflict.
    \item Our merging operator, (//), receives two patches and returns a patch possibly annotated with conflicts. We do so by matching the spines, and carefully inspecting any changes where the spines differ.
    \begin{minted}[escapeinside=!!]{haskell}
        p // q = txMap (uncurry' reconcile) $ txGCP p q
    \end{minted}
    \item Here, the \texttt{reconcile} function shall check whether the disagreeing parts are disjoint
\end{itemize}

\section{Experiments}
\begin{itemize}
    \item Merge Evaluation. We were able to solve a total of 66 conflicts automatically, amounting to 11\% of all the conflicts we encountered.
    \item Performance Evaluation. In order to evaluate the performance of our implementation we have timed the computation of the two diffs, \texttt{diff o a} and diff o b, for each merge conflict \texttt{a, o, b} in our dataset.
    \item The results were expected given that we have seen how diff x y runs in O(n + m) where n and m are the number of constructors in x and y abstract syntax trees, respectively.
    \item We see that around 90\% of our dataset falls within a one-second runtime.
    \begin{figure}[H]
        \centering
        \includegraphics[width=.8\textwidth]{plot_time_diffing.png}
        \caption{Plot of the time for diffing files}
        \label{fig:diffing_time}
    \end{figure}
    \item There are two main threats to the validity of our empirical results. 
    \begin{itemize}
        \item Firstly, we are diffing and merging abstract syntax trees, hence ignoring comments and formatting.
        \item Secondly, a significant number of developers prefer to rebase their branches instead of merging them. Therefore, we may have missed a number of important merge conflicts that are no longer recorded, as rebasing erases history.
    \end{itemize}
\end{itemize}

\section{Future Work}
\begin{itemize}
    \item One interesting direction for further work is how to control the sharing of subtrees. As it stands, the differencing algorithm will share every subtree that occurs in both the source and destination files. This can lead to undesirable behavior.
    \item Better Merge Algorithm.
    \item Extending the Generic Universe.
\end{itemize}

\section{Conclusion}
Throughout this paper we have developed an efficient type-directed algorithm for computing structured differences for a large class of algebraic datatypes, namely, mutually recursive families. This class of types can represent the abstract syntax tree of most programming languages and, hence, our algorithm can be readily instantiated to compute the difference between programs written in these languages. We have validated our implementation by computing diffs between Lua source files obtained from various repositories on GitHub; the algorithm's run-time is competitive, and even a naive merging algorithm already offers a substantial improvement over existing technology, for the former is tree-based and the latter is line-based. Together, these results demonstrate both a promising direction for further research and a novel application of the generic programming technology that is readily available in today's functional languages.