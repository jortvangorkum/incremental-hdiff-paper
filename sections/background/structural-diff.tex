\section{Background}

\subsection{An Efficient Algorithm for Type-Safe Structural Diffing}
The paper \textit{An Efficient Algorithm for Type-Safe Structural Diffing} by Victor Cacciari Miraldo and Wouter Swierstra presents an efficient datatype-generic algorithm called \textit{hdiff} to compute the difference between two values of any algebraic datatype. In particular, the algorithm readily works over the abstract syntax tree (AST) of a programming language\cite{miraldo2019efficient}.

To make the \textit{hdiff} algorithm work, an implementation of which common subtree needs to be defined. The \texttt{wcs} function is a function that when given two trees and a subtree, returns the position of the subtree inside the trees if both contain the subtree. Otherwise, the function returns nothing. An example of a naive implementation would be:

\begin{haskell}
wcs :: Tree -> Tree -> Tree -> Maybe Int
wcs s d x = elemIndex x (subtrees s !$\cap$! subtrees d) 
\end{haskell}

Here the function \texttt{subtrees} enumerates all the subtrees of a given tree. Then \texttt{elemIndex} returns the index when the subtree is found, otherwise it returns nothing. 

The paper identifies two inefficiencies using this naive implementation.
\begin{enumerate*}[label=(\Alph*)]
    \item Furthermore, enumerating all subtrees is exponential;
    \item checking trees for equality is linear in the size of the tree.
\end{enumerate*}

To improve the first inefficiency of the naive \texttt{wcs} implementation is to use cryptographic hash functions to compare the equality of the trees. To check the trees for equality in constant time the trees are decorated with a hash at every node in the tree. Then, using the precomputed hash and the root node of the given tree, the hash of a subtree is calculated in constant time.

The second inefficiency of the naive \texttt{wcs} implementation is improved by using a \texttt{Trie}\cite{brass2008trie} datastructure. Given that a \texttt{Hash} is just a \inlinehaskell{[Char]}, this makes the \texttt{Trie} datastructure the preferred choice to store the enumerated subtrees. And because the \texttt{Hash} has a constant size the \texttt{Trie} lookups are efficient and runs in amortized constant time.

