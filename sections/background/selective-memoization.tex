\section{Selective memoization}

The paper \textit{Selective Memoization} by Umut A. Acar, Guy E. Blelloch and Robert Harper\cite{acar2003selective} presents a framework for applying memoization selectively. Also, it describes what type of key issues there are with implementing memoization efficiently:
\begin{enumerate*}[label={(\Alph*)}]
    \item equality; 
    \item precise dependences and 
    \item space management.
\end{enumerate*}

For equality, the cost of having an equality test can negate the advantage of using memoization. In the paper, there are a few approaches proposed to alleviate this problem. The first is based on the equality test not having to be exact. So, for expensive equality tests, it could determine to skip the test or use a less expensive equality test. 

The second approach suggested is to ensure that there is only one copy of every value, known as a "hash consing". If there is only one copy, equality can then be implemented by comparing locations. The problem with hash consing is it demands a large amount of memory and has trouble working with the garbage collection. An alternative proposed by \Citeauthor{pugh1989incremental}\cite{pugh1989incremental}, is lazy structure sharing.\todo{Maybe add an explanation of what lazy structure sharing is}

For precise dependences
\todo[inline]{Add precise dependences}

\todo[inline]{Add space management}