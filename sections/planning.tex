\section{Timetable and Planning}
\subsection{Exploratory topics}
During the first part of the Thesis project, multiple topics are thought of that need further research/implementation in the second part of the Thesis project.

To implement the generic \texttt{merkle}, \texttt{cata} and \texttt{cataWithMap} functions, the \texttt{generics-msrop} library described in Section \ref{sec-sums-of-products} is a good candidate to use for implementing these functions. This is because it supports mutually recursive datatypes, meaning that a large group of datatypes are supported. 

In the paper \textit{Selective Memoization} in Section \ref{sec-select-mem}, there are three key issues highlighted to focus on for implementing an efficient memoization algorithm: equality, precise dependencies, and space management. The parameters for these key issues could be further researched.

In the paper \textit{Concise, Type-Safe, and Efficient Structural Diffing} in Section \ref{sec-concise-struct-diff} they describe using two equivalence relations instead of one. Using two equivalence relations could lead to more opportunities for reusing computed results. However, further research is needed on how feasible it is using two equivalence relations. 

For the data structures used for storing the incremental computation, the easiest to use would be using a \texttt{HashMap}. But, the paper \textit{An Efficient Algorithm for Type-Safe Structural Diffing} described in Section \ref{sec-efficient-struct-diff} suggest using a different data structure, the \texttt{Trie} data structure. Further research could be done in comparing the performance and memory usage of both data structures.

To update the data type and its merkelized version in an efficient manner, we only want to update the hashes of the changes and their parents. To support this, we need functionality for the developer to say where a piece of the data type has to be changed. There are two approaches to support this. The first approach is a Zipper\cite{huet1997zipper}. A Zipper is a pointer that points to a specific location in a data structure. The second approach is Lenses\cite{steckermeier2015lenses}.


\todo[inline]{Add an explanation of incremental updating Merkle tree and using Zipper or Lenses to solve it}

\newpage
\subsection{Schedule}

Priority 1 is the lowest, 5 is the highest.
\begin{table}[H]
    \setlength{\tabcolsep}{8pt}
    \centering
    \small
    \bigskip
    \begin{tabular}{|c|l|l|}
        \hline
        Category & Work & Priority \\
        \hline
        \multirow{3}{7em}{Implementation} & Implementing Generic CataMerkle library using \texttt{generics-msrop} & 5 \\
         & Implementing incremental update merkle tree & 5 \\
         & Implementing tests & 4 \\
        \hline
        \multirow{6}{7em}{Experiments} & Creating benchmarks & 5 \\
         & Experiment using different parameters for space management & 4 \\
         & Experiment using different data structures & 4 \\
         & Experiment using real-world data & 3 \\
         & Experiment using different parameters for equality & 2 \\
         & Experiment using different parameters for precise dependencies & 1 \\
        \hline
    \end{tabular}
\caption{Category priority list}
\label{table:priorities}
\end{table}

\begin{table}[H]
\setlength{\tabcolsep}{8pt}
\centering
\small
\bigskip
\begin{tabular}{|l|l|l|}
    \hline
    Week & Date & Category \\
    \hline
    Week 1 - 4 & 28 feb - 25 mar & Implementation \\
    \hline
    Week 5 - 10 & 28 mar - 06 may & Experiments \\
    \hline
    Week 11 - 13 & 08 mar - 13 may & Writing \\
    \hline
    Week 14 - 17 & 16 may - 3 jun & Feedback \\
    \hline
    Week 18 - 19 & 6 jun - 17 jun & Time Left (Vacation/Overdue Work) \\
    \hline
    Week 20 & 20 jun - 24 jun & Finalize \\
    \hline
    Week 21 & 27 jun - 1 jul & Vacation \\
    \hline
    Week 27 & 08 aug - 12 aug & Submission \\
    \hline 
    Week 28 & 15 aug & End Date Research Project  \\
    \hline
\end{tabular}
\caption{Planning per category}
\label{table:planning}
\end{table}